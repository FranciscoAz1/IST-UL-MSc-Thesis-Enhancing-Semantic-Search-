% #############################################################################
% Abstract Text
% !TEX root = ../main.tex
% #############################################################################
% reset acronyms
\acresetall
% use \noindent in firts paragraph
\noindent  
This thesis focuses on enhancing semantic search capabilities in corporate environments, where the exponential growth of documents makes management, retrieval, and processing increasingly challenging. Company repositories often include rich metadata, complex workflows, and sensitive associations that require more than traditional keyword-based retrieval. To address these challenges, this work explores recent advances in \ac{NLP}, focusing on \ac{RAG} and its extension into \ac{AAI}.


The project investigates how vector databases, combined with large language models, can serve as the foundation for scalable retrieval-augmented systems. Retrieval strategies were first studied in isolation, highlighting how document structures and metadata can be leveraged for more effective context retrieval. Building on this foundation, the research then explored Agentic RAG, where autonomous agents dynamically orchestrate retrieval across multiple datasets, enabling reasoning over distributed information sources.


The work was carried out in collaboration with Edoclink, a company project aiming to modernize document management through AI. The thesis provides a walkthrough of the research process, showing how concepts from retrieval augmentation and Agentic AI can be applied in practice, and how such approaches align with the company’s goals. In doing so, it contributes both a technical study of emerging methods and an applied perspective on their potential for enterprise environments.