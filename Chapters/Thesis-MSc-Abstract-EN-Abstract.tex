% #############################################################################
% Abstract Text
% !TEX root = ../main.tex
% #############################################################################
% reset acronyms
\acresetall
% use \noindent in firts paragraph
\noindent  The focus of this thesis is to enhance semantic search capabilities within organized company environments, where the number of documents tends to grow rapidly, making it increasingly difficult to manage, retrieve, process, and monitor changes effectively. This thesis will account for documents with rich metadata, complex structures, progressive stages, workflows, entity associations, digital signatures, and any other meta-data used for document organization. This study seeks methods to semi-automate these time-consuming tasks using \ac{NLP}.

The adoption of open-source pre-trained language models has revolutionized \ac{NLP}, proving particularly effective for tasks involving natural language understanding and information retrieval. This work proposes leveraging the capabilities of open-source pre-trained \ac{LLM}, \ac{RAG}, and \ac{VD}, then enhance their utility, efficiency and precision with techniques such as llm profiling, fine-tuning, and context management to address these challenges. Such advancements enable the development of powerful AI assistants capable of natural language interaction, performing tasks like searching the internet for relevant information, refining and enhancing written documents, or, in company environments, securely accessing and retrieving data from private repositories. These innovations aim to bridge the gap between machine and human communication, making interactions more seamless and intuitive.

As recommended by Edoclink, the sponsoring company, the system is designed for seamless integration across different environments, for example processing emails, documents, and workflows to the vector database. The goal is to enhance usability and broaden the effectiveness of semantic search by expanding the accessible context and information.

The solution consists of an automated pipeline that processes raw inputs—such as PDFs—and stores extracted information in a \ac{VD}. This enables an optimal and efficient \ac{QA} interface by organizing information for fast retrieval. Advanced algorithms, like  knowledge graphs that store information with links to relevant information, nearest neighbor search, that query only the most relevant data,  A \ac{LLM} then summarizes the results to support quick and effective understanding.
