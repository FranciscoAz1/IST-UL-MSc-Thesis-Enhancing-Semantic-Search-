% #############################################################################
% Abstract Text
% !TEX root = ../main.tex
% #############################################################################
% reset acronyms
\glsresetall
% use \noindent in firts paragraph
\noindent  
This thesis designs and implements a retrieval‑augmented, agentic semantic search system for corporate document management, developed with Edoclink. Corporate repositories contain rich metadata and organizational structure that traditional keyword search cannot exploit. We explore a newsroom scenario where journalists query documents to fact‑check articles; the system retrieves relevant content, follows cross‑references for programmable multi‑hop retrieval, and synthesizes cited answers for efficient verification.

We leverage advances in \glsxtrfull{NLP}, centering on \glsxtrfull{RAG} and \glsxtrfull{AAI}. The pipeline (i) ingests and normalizes heterogeneous documents, (ii) indexes content in a \glsxtrfull{VD} with metadata‑aware schemas supporting explicit cross-references, and (iii) uses \glsxtrfullpl{LLM} orchestrated via \glsxtrfull{PE} to ground answers in retrieved evidence. We study retrieval in isolation—structure‑aware chunking, metadata filtering, and hybrid search—then implement Agentic RAG, where agents perform query analysis, collection selection, iterative retrieval, and cited synthesis.
A key contribution is our \glsxtrfull{MCP} server (Go and TypeScript) exposing Weaviate's schema-aware query and cross-reference traversal as typed tools, enabling agents to navigate organizational structures programmatically. We demonstrate functionality through a newsroom scenario with multi-hop queries. We validate retrieval on synthetic enterprise documents (1,000 files, 300 QA pairs) and temporal knowledge bases. Using Youden-optimized thresholds, our agentic system achieved 60.4\% retrieval accuracy (versus 13.8\% naive RAG), with 61\% answers classified correct (versus 12–19\%)—a 4.4× improvement at 27.6× cost, demonstrating favorable trade-offs for precision-critical applications.