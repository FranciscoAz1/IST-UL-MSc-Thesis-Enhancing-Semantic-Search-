% #############################################################################
% Abstract Text
% !TEX root = ../main.tex
% #############################################################################
% reset acronyms
\glsresetall
% use \noindent in first paragraph
\noindent  
This thesis designs and implements a retrieval‑augmented, agentic semantic search system for corporate document management, developed with Edoclink. Corporate repositories contain rich metadata and organizational structure that traditional keyword search cannot exploit. We explore a newsroom scenario where journalists query documents to fact‑check articles; the system retrieves relevant content, follows cross‑references for programmable multi‑hop retrieval, and synthesizes cited answers for efficient verification.

We leverage advances in \glsxtrfull{NLP}, centering on \glsxtrfull{RAG} and \glsxtrfull{AAI}. The pipeline (i) ingests and normalizes heterogeneous documents, (ii) indexes content in a \glsxtrfull{VD} with metadata‑aware schemas supporting explicit cross-references, and (iii) uses \glsxtrfullpl{LLM} orchestrated via \glsxtrfull{PE}, to ground answers in retrieved evidence. We study retrieval in isolation—structure‑aware chunking, metadata filtering, and hybrid search—then implement Agentic RAG, where agents perform query analysis, collection selection, iterative retrieval, and cited synthesis.
A key contribution of this work is the development of an \glsxtrfull{MCP} server, implemented in Go and TypeScript, that exposes Weaviate's schema-aware querying and cross-reference traversal as typed tools—enabling agents to navigate organizational structures programmatically. The system's capabilities are demonstrated through a newsroom scenario featuring multi-hop queries. Retrieval performance is validated on synthetic enterprise information (1,000 files, 300 QA pairs) as well as on temporal knowledge bases. Using Youden-optimized thresholds, the agentic system achieved a 60.4\% retrieval accuracy (compared to 13.8\% naive \gls{RAG} baseline), and correctly classified 61\% of answers (versus 19\% naive \gls{RAG} baseline), representing a 27.6× cost increase—demonstrating a favorable trade-off for precision-critical applications.