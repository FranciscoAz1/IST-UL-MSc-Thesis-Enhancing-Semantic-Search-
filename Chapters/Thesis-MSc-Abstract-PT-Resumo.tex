% #############################################################################
% RESUMO em Português
% !TEX root = ../main.tex
% #############################################################################
% use \noindent in firts paragraph
% reset acronyms
\glsresetall
\noindent
Esta tese concebe e implementa um sistema de pesquisa semântica com recuperação aumentada e baseado em agentes para gestão de documentos corporativos, desenvolvido com a Edoclink. Os repositórios corporativos contêm metadados e estrutura organizacional que a pesquisa tradicional por palavras-chave não explora. Exploramos um cenário jornalístico onde jornalistas consultam documentos para verificar factos; o sistema recupera conteúdo relevante, segue referências cruzadas para recuperação multi-salto programável, e sintetiza respostas citadas para verificação eficiente.

Aproveitamos avanços em \gls{NLP}, centrando em \gls{RAG} e \gls{AAI}. O pipeline (i) ingere e normaliza documentos heterogéneos, (ii) indexa conteúdo numa \gls{VD} com esquemas sensíveis aos metadados suportando referências cruzadas explícitas, e (iii) utiliza \glspl{LLM} orquestrados via \gls{PE} para fundamentar respostas em evidência recuperada. Estudamos recuperação isoladamente (segmentação estruturada, filtragem por metadados, pesquisa híbrida), depois implementamos \gls{ARAG}, onde agentes realizam análise de perguntas, seleção de coleções, recuperação iterativa, e síntese citada.
Uma contribuição fundamental é o servidor \gls{MCP} (Go e TypeScript) que expõe consultas sensíveis ao esquema e travessia de referências do Weaviate como ferramentas tipificadas, permitindo agentes navegarem estruturas organizacionais programaticamente. Demonstramos funcionalidade através de cenário jornalístico com consultas multi-salto. Validamos recuperação em documentos empresariais sintéticos (1.000 ficheiros, 300 pares QA) e bases temporais. Com limiares Youden, o sistema alcançou 60,4\% precisão de recuperação (versus 13,8\% RAG básico), com 61\% respostas corretas (versus 12–19\%)—melhoria 4,4× com custo 27,6×, demonstrando equilíbrio favorável para aplicações críticas.





