% #############################################################################
% RESUMO em Português
% !TEX root = ../main.tex
% #############################################################################
% use \noindent in firts paragraph
% reset acronyms
\glsresetall
\noindent
Esta tese concebe e implementa um sistema de pesquisa semântica com recuperação aumentada e baseado em agentes para gestão de documentos corporativos, desenvolvido com a Edoclink. Os repositórios corporativos contêm metadados ricos e fluxos de trabalho que vão além da pesquisa por palavras‑chave. Por exemplo, em verificações de conformidade regulamentar, ao redigir um novo contrato o sistema recupera os regulamentos mais recentes e raciocina sobre eles para determinar se o rascunho está em conformidade, e identificar quais cláusulas podem requerer revisão, com citações para as passagens de origem.

Aproveitamos avanços em \gls{NLP}, centrando-se em \gls{RAG} e na sua extensão para \gls{AAI}. O pipeline do sistema (i) ingere e normaliza documentos heterogéneos, (ii) indexa o conteúdo numa \gls{VD} com esquemas sensíveis aos metadados, e (iii) utiliza \glspl{LLM} orquestrados via \gls{PE} para fundamentar as respostas em evidência recuperada. Estudamos a recuperação de forma isolada, (segmentação estruturada/hierárquica, filtragem por metadados e pesquisa híbrida esparsa e densa),e depois implementamos \gls{ARAG}, onde agentes especializados realizam análise da pergunta, seleção de coleções, recuperação iterativa, uso de ferramentas e síntese com citação através de fontes distribuídas.
Validamos em conjuntos de dados e tarefas empresariais representativos, alinhados com os casos de uso da Edoclink, documentando a concepção e a avaliação(qualidade da resposta, rastreabilidade, latência e custo) e fornecendo um guia implementável para contextos empresariais.

Sob limiares otimizados por Youden num benchmark \gls{QA} sintético de 300 questões sobre uma empresa, a métrica Token Recall classificou 61\% das respostas como corretas para o sistema baseado em agentes, face a 12–19\% nos baselines básicos.





