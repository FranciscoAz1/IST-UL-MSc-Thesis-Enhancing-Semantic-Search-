% #############################################################################
% This is Chapter 6
% !TEX root = ../main.tex
% #############################################################################
% Change the Name of the Chapter i the following line
\fancychapter{Conclusion}
\cleardoublepage%
% The following line allows to ref this chapter
\label{chap:conclusion}%


% #############################################################################
\section{Conclusions}

This thesis investigated how to enhance semantic search by combining retrieval with large language models, structure-aware system design, and agentic workflows. The results show that well-crafted prompting, effective reasoning strategies, and careful grounding in external knowledge can improve both retrieval and answer quality.

A design was proposed for structure-aware systems that leverage an organization's inherent structure—such as cross-references, collection descriptions, and metadata—to guide retrieval. Although large-scale evaluation was outside the scope, the analysis suggests that, in company settings, such structural signals can meaningfully improve retrieval precision.

Prompt-engineering recipes for document analysis were also explored (e.g., metadata extraction, organization suggestions, targeted entity extraction, and summarization). While initial results were modest due to compute and budget constraints, the approach is promising and likely to improve with more efficient optimization, stronger models, and better tuning or few-shot prompting for the application-specific schema.

\gls{AAI} systems—incorporating planning, retrieval, observation, and reflection—were compared to a naive retrieval-augmented generation \gls{RAG} baseline. Across the tasks considered, the agentic approach produced more accurate and relevant answers, underscoring the value of reasoning frameworks and prompt engineering for making information more intuitively accessible.

From an engineering perspective, extensive use was made of the Weaviate vector database, which enabled rapid experimentation with configurations and provides a clear path to scalability, reproducibility, and addition of \glspl{ACL} for corporate environments. An \gls{MCP} server was also built to interface with Weaviate, allowing standard \gls{MCP} clients (e.g., agents and chatbots) to interact with the knowledge base (including traversal).

Collectively, this thesis advances \gls{RAG} pipeline techniques for extracting structured information from unstructured documents via systematic prompt engineering. Proposes a structure-aware retrieval design that leverages organizational signals, and delivers an \gls{MCP} server that integrates with a scalable, AI-enabled vector database (Weaviate) to connect with chatbots and tool-using agents—together providing a practical foundation for building more effective and user-friendly information-retrieval systems across domains. Taken together, the results in Chapter~\ref{chap:results} also crystallize best practices and configuration guidelines for the schema-aware agent over cross-references in Weaviate—when to favor agentic planning, how to set hybrid search/top-$k$, when to enable multi-collection search, and when to apply extractive summarization—so practitioners can deploy the MCP–Weaviate stack in a robust, cost-aware, and accuracy-oriented manner.
% #############################################################################
\section{System Limitations and Future Work}
The primary limitations of this work were computational resources, financial budget, and data availability. Our local compute was sufficient to run only small open \glspl{LLM}; evaluating stronger models with longer context windows required using hosted APIs. Per-token pricing for hosted \glspl{LLM} (e.g., OpenAI, Google Gemini) kept day-to-day iteration affordable but made large-scale ablations and benchmarking costly, which in turn limited the breadth of experiments, the size and complexity of models we could test, and the context-window lengths we could routinely exercise. These constraints also curtailed our evaluation of document-analysis pipelines: they were comparatively expensive to run and yielded only modest gains in our setting. Finally, the lack of a publicly available dataset tailored to structure-aware retrieval hindered a comprehensive end-to-end assessment, and the specific model choices narrow the generality of our conclusions.

By design, this thesis does not propose new \gls{ML} or \gls{LLM} architectures. It assumes that contemporary \glspl{LLM} and embedding models are already sufficiently capable for many enterprise scenarios and will continue to improve primarily through scale (more parameters and longer context windows). Accordingly, the focus is on \gls{RAG} and prompt-engineering techniques that exploit long-context models to deliver better answers—especially for non-expert users who may not know how to prompt \glspl{LLM} effectively.

Future work should:
\begin{itemize}
	\item evaluate stronger base and instruction-tuned models, with longer context windows, on larger and more diverse datasets;
	\item develop and release a public benchmark for structure-aware retrieval that captures cross-references, hierarchies, and organizational context, with human-annotated relevance judgments;
	\item systematically compare prompting and reasoning strategies 
	
	(e.g., self-consistency, tree-of-thought, \textsc{ReAct}) across domains and tasks;
	\item integrate additional external knowledge sources and more robust grounding (e.g., entity linking, schema alignment, knowledge graphs, retrieval-augmented planning), and evaluate hallucination/faithfulness;
	\item study user experience with non-experts via controlled studies, measuring usability, transparency, latency, cost, and confidence calibration;
	\item quantify cost–quality trade-offs and throughput under realistic workloads, including caching, reranking, and distillation for efficiency.
\end{itemize}

These directions would enable a fuller assessment of the architecture and its benefits in real-world deployments.



