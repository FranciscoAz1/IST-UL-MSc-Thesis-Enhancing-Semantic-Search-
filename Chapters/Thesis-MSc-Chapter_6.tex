% #############################################################################
% This is Chapter 6
% !TEX root = ../main.tex
% #############################################################################
% Change the Name of the Chapter i the following line
\fancychapter{Conclusion}
\cleardoublepage%
% The following line allows to ref this chapter
\label{chap:conclusion}%


% #############################################################################
\section{Conclusions}
In this thesis, we investigated how to enhance semantic search by combining retrieval with large language models, structure-aware system design, and agentic workflows. Our results show that well-crafted prompting, lightweight reasoning strategies, and careful grounding in external knowledge can improve both retrieval and answer quality.

We proposed a design for structure-aware systems that leverage an organization's inherent structure—such as cross-references, hierarchies, and metadata—to guide retrieval. Although large-scale evaluation was outside our scope, our analysis suggests that, in enterprise settings, such signals can meaningfully improve retrieval precision.

We also explored prompt-engineering recipes for document analysis (e.g., metadata extraction, organization suggestions, targeted entity extraction, and summarization). However, these pipelines yielded limited gains relative to cost in our setting and are not a core contribution.

We compared agentic AI systems—incorporating planning, retrieval, observation, and reflection—to a naive retrieval-augmented generation (RAG) baseline. Across our tasks, the agentic approach produced more accurate and relevant answers, underscoring the value of reasoning frameworks and prompt engineering for making information more accessible, even to non-expert users.

From an engineering perspective, we made extensive use of the Weaviate vector database, which enabled rapid experimentation with configurations and provides a clear path to scalability and reproducibility. We also built an MCP server that interfaces with Weaviate, allowing standard MCP clients (e.g., agents and chatbots) to interact with the knowledge base.

Collectively, these contributions advance the practice of semantic search, prompt design, and agentic systems, and provide a practical foundation for building more effective and user-friendly information-retrieval systems across domains.
% #############################################################################
\section{System Limitations and Future Work}
The primary limitations of this work were computational resources and data availability, which constrained the breadth of our experiments and the size and complexity of models and datasets we could evaluate. These constraints also limited the evaluation of document-analysis pipelines, which were comparatively costly and produced modest gains in our setting. Moreover, our choice of models restricted the generality of the conclusions. Finally, the absence of a publicly available dataset tailored to structure-aware retrieval prevented a comprehensive end-to-end evaluation of the proposed architecture.

Future work should:
\begin{itemize}
	\item evaluate stronger models and architectures on larger, more diverse datasets;
	\item develop and release a benchmark for structure-aware retrieval that captures cross-references and organizational context;
	\item systematically compare prompting and reasoning strategies across domains and tasks;
	\item integrate additional external knowledge sources and more robust grounding techniques (e.g., entity linking, schema alignment, retrieval-augmented planning);
	\item study user experience with non-experts through controlled user studies, focusing on usability, transparency, latency, and cost.
\end{itemize}

These directions would enable a fuller assessment of the architecture and its benefits in real-world deployments.



